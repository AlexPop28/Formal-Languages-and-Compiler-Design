% Created 2023-10-15 Sun 12:31
% Intended LaTeX compiler: pdflatex
\documentclass[11pt]{article}
\usepackage[utf8]{inputenc}
\usepackage[T1]{fontenc}
\usepackage{graphicx}
\usepackage{longtable}
\usepackage{wrapfig}
\usepackage{rotating}
\usepackage[normalem]{ulem}
\usepackage{amsmath}
\usepackage{amssymb}
\usepackage{capt-of}
\usepackage{hyperref}
\author{Alex Pop}
\date{\today}
\title{Lab 1a assignment}
\hypersetup{
 pdfauthor={Alex Pop},
 pdftitle={Lab 1a assignment},
 pdfkeywords={},
 pdfsubject={},
 pdfcreator={Emacs 29.1 (Org mode 9.6.6)}, 
 pdflang={English}}
\begin{document}

\maketitle
\tableofcontents

\section{P1: max of 3 integers}
\label{sec:orgd8830de}
\begin{verbatim}
int a;
int b;
int c;
a = read_int();
b = read_int();
c = read_int();
int ans;
ans = a;
if (b > ans) {
  ans = b
}
if (c > ans) {
  ans = c;
}
print_int(ans);
\end{verbatim}

\section{P2: primality check}
\label{sec:org557d9fb}
\begin{verbatim}
int n;
n = read_int();
int prime;
prime = 1;
int d;
d = 2;
while (d < n) {
  if (n % d == 0) {
    prime = 0;
  }
  d = d + 1;
}
if (prime == 0) {
  print_str("not prime")
} else {
  print_str("prime")
}
\end{verbatim}

\section{P3: sum of n numbers}
\label{sec:org2329c30}
\begin{verbatim}
int n;
n = read_int();
int i;
i = 0;
int sum;
sum = 0;
while (i < n) {
  int x;
  x = read_int();
  sum = sum + x;
}
print_int(sum);
\end{verbatim}

\section{P1err: sum of n numbers with 2 lexical errors}
\label{sec:org68ba7ab}
\begin{verbatim}
int _n;
_n = read_int(); // wrong comment
int i;
i = 0;
int sum;
sum = 0;
while (i < _n) {
  int x;
  x = read_int();
  sum = sum + x;
}
print_int(sum);

\end{verbatim}
On the first line, \texttt{int \_n;}, there is a lexical error since \texttt{\_n} cannot be classified as a token.
On the second line, \texttt{\_n = read\_int(); // wrong comment}, there is another lexical error since \texttt{/} is not part of the alphabet.
\end{document}